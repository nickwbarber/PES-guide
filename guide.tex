\documentclass[a4paper,12pt]{article}

\usepackage{verbatim}
\usepackage{booktabs} 
\usepackage{tabularx}
\usepackage{caption}

\captionsetup[table]{name=Example Set}

\renewcommand{\familydefault}{\rmdefault}
\renewcommand*\rmdefault{bch}

\newcommand{\ra}[1]{\renewcommand{\arraystretch}{#1}}

\newcounter{magicrownumbers}
\newcommand\rownumber{\refstepcounter{magicrownumbers}\arabic{magicrownumbers}}

\begin{document}

\title{Explanatory Style Annotation Guide}
\author{HiLT Lab}
\date{\today\\v1.0}
\maketitle

\tableofcontents
\break

Research suggests that the way that people explain things that happen in their lives can be revealing.\footnote{\label{seligman89}Schulman, P., Castellon, C., \& Seligman, M. (1989). Assessing explanatory style: The content analysis of verbatim explanations and the attributional style questionnaire. Behavior Research and Therapy, 27(5), 505-512} % try to find a more appropriate citation; preferably one that doesn't talk as much about the questionnaire (or perhaps just mention the questionnaire)
A consistently pessimistic \emph{explanatory style}, for example, can help indicate depression.
The ability to detect these explanations, and consequently describe a person's explanatory style, is therefore a valuable skill for therapy-minded Companionbots since it lets them know when and about what to start a therapeutic dialogue.

\section{Event Attribution Units}
In order to describe a person's explanatory style, we need a sample of how they explain things that happen to them (particularly things about which the person has a non-neutral opinion). % briefly describe CAVE analysis and difference from questionnaire
This sample is made up of a set of \emph{event-attribution units}\footnote{\label{CAVE}appendix of CAVE paper, with guidelines} (EAUs) and some descriptive features about them. % citation needed
EAUs are the conjunction of an \emph{event} and an \emph{attribution} that explains the existence of the event.
 

\subsection{Events}

Seligman provides a fairly comprehensive definition of what is meant by \emph{event} in his annotation guidelines: % cite, revise "his"

\begin{quote}
    An event is defined as any stimulus that occurs in an individual's environment or within that individual (e.g. thoughts or feelings) that has a good or bad effect from the individual's point of view.
    Events can be mental (e.g. I was afraid), social (e.g. I got a pay raise) or physical (e.g. I got in a car accident).
    Events should be unambiguously good or bad from the individual's point of view and may occur in the past, present or hypothetical future.
    Events that have good and bad elements, neutral events or events that do not affect the \emph{S} [the individual] should not be extracted.
\end{quote}

While we do follow this definition as closely as possible, we find that it is easier to reproduce similarly formed \emph{events} in our own verbatim transcripts by adding a few conditions to the definition given above:
\begin{enumerate}
    \item \emph{Events} must be complete sentences.
    \item \emph{Events} must \emph{propose} an \emph{event}.
    \item The linguistic form of an \emph{event} must be similar to that of a simplified linguistic expression of the \emph{proposition} of that \emph{event}. % give examples as reasons
    \item \emph{Events} must be the minimal length necessary to satisfy all other given conditions.
\end{enumerate}

\subsection{Attributions}
\subsection{Summary}

He then goes on to give examples of good \emph{event-attribution units}: % reproduce examples
\setcounter{magicrownumbers}{0}
\begin{table}[h!]\centering
    \ra{1.3}
    \caption{Good EAUs.}
    \label{table1}
    \noindent\begin{tabularx}{\textwidth}{llX}
        \toprule
        & Event & Attribution \\
        \midrule
        \rownumber & I didn't do well on my exam & because I didn't sleep well last night. \\
        \rownumber\label{ex_grad_prog} & I haven't been sleeping well & because I'm worried about getting into a good graduate program. \\
        \bottomrule
    \end{tabularx}
\end{table}

I want to refer to example \ref{table1}.\ref{ex_grad_prog}

\section{Analyzing EAUs}
\subsection{Personal v. External}
\subsection{Permanent v. Temporary}
\subsection{Pervasive v. Specific}
\subsection{Summary}

\section{Annotating Event-Attribution Units and their Features}
\subsection{Phase 1}
\subsection{Phase 2}
\subsection{Phase 3}


\end{document}
