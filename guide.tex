\documentclass[a4paper,12pt]{article}
\begin{document}

\title{Explanatory Style Annotation Guide}
\author{HiLT Lab}
\maketitle


Research suggests that the way the people explain things that happen in their lives can be revealing.\footnote{\label{seligman89}Schulman, P., Castellon, C., \& Seligman, M. (1989). Assessing explanatory style: The content analysis of verbatim explanations and the attributional style questionnaire. Behavior Research and Therapy, 27(5), 505-512} % try to find a more appropriate citation; preferably one that doesn't talk as much about the questionnaire (or perhaps just mention the questionnaire)
A consistently pessimistic \emph{explanatory style}, for example, can help indicate depression.
The ability to detect these explanations, and consequently describe a person's explanatory style, is therefore a valuable skill for therapy-minded Companionbots since it lets them know when and about what to start a therapeutic dialogue.

\section{Event Attribution Units}
In order to describe a person's explanatory style, we need a sample of how they explain things. % briefly describe CAVE analysis and difference from questionnaire
This sample is made up of a set of \emph{event-attribution units}\footnote{\label{CAVE}CAVE paper, and their guidelines} (EAUs) and some descriptive features about them.
EAUs are the conjunction of an \emph{event} and an \emph{attribution} that explains the existence of the event.
 

\subsection{Events}
As stated by Seligman, events are defined as “any stimulus that occurs in an individual’s environment or
within that individual (e.g. thoughts or feelings) that has a good or bad effect from the individual’s point of
view.” He further elaborates that events may be mental, social, or physical. An example of a mental event is
“I was afraid.” An example of a social event is “I got a pay raise.” An example of a physical event is “I got in a
car accident.” Though all of these examples are stated in the simple past, events may occur in the past (“I
was afraid”), present (“I am afraid”), or hypothetical future (“I may become afraid”).
A critical element of defining events is their polarity. Events must be \emph{unambiguously positive or negative
in terms of its effects from the perspective of the speaker.} If the speaker has conflicted feelings about
the event, it should not be annotated. If the speaker feels that the event is neutral (neither positive or
negative) then it should not be annotated. If the event does not impact the speaker, it should not be
annotated. Remember that what is important is how the speaker perceives the event. As an annotator, you
may think that an event is positive, but the client could feel negatively or conflicted about it (e.g. quitting
smoking).

\subsection{Attributions}
\subsection{Summary}

\section{Analyzing EAUs}
\subsection{Personal v. External}
\subsection{Permanent v. Temporary}
\subsection{Pervasive v. Specific}
\subsection{Summary}

\section{Annotating Events and their Attributions}
\subsection{Phase 1}
\subsection{Phase 2}
\subsection{Phase 3}


\end{document}
