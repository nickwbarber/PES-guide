\documentclass[a4paper,12pt]{article}

\usepackage{verbatim}
\usepackage{booktabs}
\usepackage{nameref}
\usepackage{tabularx}
\usepackage{caption}
\usepackage{gb4e}

\noautomath

\captionsetup[table]{}

\renewcommand{\familydefault}{\rmdefault}
\renewcommand*\rmdefault{bch}

\newcommand{\ra}[1]{\renewcommand{\arraystretch}{#1}}

\newcounter{magicrownumbers}
\newcommand\rownumber{\refstepcounter{magicrownumbers}\arabic{magicrownumbers}}

\begin{document}

\title{Explanatory Style Annotation Guide -- DRAFT}
\author{HiLT Lab}
\date{\today\\v0.6.1}
\maketitle

\tableofcontents
\break

Research suggests that the way that people explain things that happen in their lives can be revealing.\footnote{\label{seligman89}Schulman, P., Castellon, C., \& Seligman, M. (1989). Assessing explanatory style: The content analysis of verbatim explanations and the attributional style questionnaire. Behavior Research and Therapy, 27(5), 505-512} % TODO: try to find a more appropriate citation; preferably one that doesn't talk as much about the questionnaire (or perhaps just mention the questionnaire)
A consistently pessimistic \emph{explanatory style}, for example, can help indicate depression.
The ability to detect these explanations, and consequently describe a person's explanatory style, is therefore a valuable skill for therapy-minded Companionbots since it lets them know when and about what to start a therapeutic dialogue.

\section{Event Attribution Units (EAUs)}
In order to describe a person's explanatory style, we need a sample of how they explain things that happen to them (particularly things about which the person has a non-neutral opinion). % TODO: briefly describe CAVE analysis and difference from questionnaire
This sample is made up of a set of \emph{event-attribution units}\footnote{\label{CAVE}appendix of CAVE paper, with guidelines} (EAUs) and some descriptive features about them. % TODO: citation needed
EAUs are the conjunction of an \emph{event} and an \emph{attribution} that explains the existence of the event.


\subsection{Events}

Seligman provides a definition of what is meant by \emph{event} in his annotation guidelines: % TODO: cite, revise "his"

\begin{quote}
    An event is defined as any stimulus that occurs in an individual's environment or within that individual (e.g. thoughts or feelings) that has a good or bad effect from the individual's point of view.
    Events can be mental (e.g. I was afraid), social (e.g. I got a pay raise) or physical (e.g. I got in a car accident).
    Events should be unambiguously good or bad from the individual's point of view and may occur in the past, present or hypothetical future.
    Events that have good and bad elements, neutral events or events that do not affect the \emph{S} [the individual] should not be extracted.
\end{quote}

While we do attempt to follow this definition as closely as possible, we find that it is easier to reproduce similarly formed \emph{events} from our own transcripts by adding several conditions to our annotations (\S \ref{conditions}) to those given in Seligman's guidelines.

\pagebreak
\subsection{Attributions}
\subsection{Examples}

(...) He then goes on to give examples of good \emph{event-attribution units}: % TODO: reproduce examples
\setcounter{magicrownumbers}{0}
\begin{table}[h!]\centering
    \ra{1.3}
    \caption{Good EAUs.}
    \label{table1}
    \noindent\begin{tabularx}{\textwidth}{llX}
        \toprule
        & Event & Attribution \\
        \midrule
        \rownumber & I didn't do well on my exam & because I didn't sleep well last night. \\
        \rownumber\label{ex_grad_prog} & I haven't been sleeping well & because I'm worried about getting into a good graduate program. \\
        \bottomrule
    \end{tabularx}
\end{table}

%I want to refer to example \ref{table1}.\ref{ex_grad_prog} %example_graduate_program

\section{Analyzing EAU dimensions}
    \subsection{Personal v. External}
    \subsection{Permanent v. Temporary}
    \subsection{Pervasive v. Specific}
    \subsection{Examples}

\pagebreak
\section{Annotating Event-Attribution Units and their Features}

    \subsection{Annotation Phases}
    \subsubsection{Phase 1: Event spans}
    \paragraph{Individual Annotation}
        \begin{enumerate}
            \item At least two annotators annotate the most appropriate text span of each event that they identify as part of an EAU, labeling each as type \texttt{Event}.
        \end{enumerate}
    \paragraph{Consensus Annotation}
        \begin{enumerate}
            \item Annotators discuss each given annotation and copy to the consensus set those which they agree meet all given annotation conditions, including the existence of an associated attribution.
                \begin{enumerate}
                    \item Annotators may decide to include any appropriate annotation (whether it be made during individual annotation or not) in the consensus set that they may happen to notice during discussion.
                \end{enumerate}
        \end{enumerate}


    \subsubsection{Phase 1B: Event polarity}
    \paragraph{Individual Annotation}
        \begin{enumerate}
            \item One annotator annotates the polarity of each \texttt{Event} given in the consensus set from Phase 1 as either \texttt{Positive} or \texttt{Negative} as a feature of the annotation.
        \end{enumerate}

    \subsubsection{Phase 2: Attribution spans}
    \paragraph{Individual Annotation}
        \begin{enumerate}
            \item For every \texttt{Event} annotation from Phase 1B, two annotators individually annotate the text span of every associated attribution as type \texttt{Attribution}.
                \begin{enumerate}
                    \item Provide the ID number of the associated event as a feature, \texttt{Caused\_Event}.
                \end{enumerate}
        \end{enumerate}
    \paragraph{Consensus Annotation}
        \begin{enumerate}
            \item Annotators discuss each given annotation and copy to the consensus set those for which they agree meet all given annotation conditions.
                \begin{enumerate}
                    \item Annotators may decide to include any appropriate annotation (whether it be made during individual annotation or not) in the consensus set that they may happen to notice during discussion.
                \end{enumerate}
        \end{enumerate}

    \subsubsection{Phase 3: Attribution dimensionality}
    \paragraph{Individual Annotation}
    
        \begin{enumerate}
            \item For every \texttt{Attribution} within the given consensus set, at least two annotators annotate values for each of the three dimensions, \texttt{Personal--External}, \texttt{Permanent--Temporary}, \texttt{Pervasive--Specific}, as features, using a digit between \texttt{1} and \texttt{7}.
        \end{enumerate}
    \paragraph{Consensus Annotation}
        \begin{enumerate}
            \item Annotators discuss each given annotation and copy to the consensus set those which they agree meet all given annotation conditions.
                \begin{enumerate}
                    \item Annotators may decide to include any appropriate annotation (whether it be made during individual annotation or not) in the consensus set that they may happen to notice during discussion.
                \end{enumerate}
        \end{enumerate}

    \subsection{Annotation span conditions} \label{conditions}
        Some definitions:
        \begin{description}
            \item[propose, \emph{v.}] to make a statement about something which could be evaluated as either \emph{true} or \emph{false},
                e.g. \emph{a bell rang repeatedly}, but not just \emph{a bell} or \emph{to ring} or \emph{repeatedly}.
            \item[corefer, \emph{v.}] to share a single reference,
                e.g. \emph{Paul} and \emph{He} in ``\emph{Paul} had a great time" and ``\emph{He}'s always in a good mood."
            \item[backchannel, \emph{v.}] to affirm active listening, e.g. ``okay," ``uh-huh," ``I'm listening," etc.
            \item[turn \emph{n.}] in conversation, an uninterrupted span of speech by a single person (excepting any \emph{backchannel}).
        \end{description}
    \subsubsection{Events}
    For every \textbf{\emph{event}} annotation:
        \begin{enumerate}
            \item The \emph{event} must have been mentioned during the given conversation.
            \item The event must have been or otherwise be very probable to have been evaluated as either good or bad by subject in terms of its effects on the subject.
            \item The text span that expresses the \emph{event} must be in the form of a complete sentence or otherwise unambiguously \emph{corefer} with a complete sentence present in the transcript expressing such event when no other non-pronoun mention of the event appears within the same \emph{turn} as the otherwise most appropriate event mention.
            \item The text span that expresses the \emph{event} must primarily \emph{propose} that event. %...", and nothing else."? % TODO: give examples as reasons
            \item Experience of the \emph{event} by the participant must not rely solely on a generic statement.
                e.g. not ``Teachers often feel stressed'' even if the individual is in fact a teacher.
            \item The text span expressing the \emph{event} must be the minimal length necessary to satisfy all other conditions.
        \end{enumerate}


    \subsubsection{Attributions}
    For every \textbf{\emph{attribution}} annotation:
        \begin{enumerate}
            \item The \emph{attribution} must have been mentioned during conversation.
            \item The text span that expresses the \emph{attribution} must primarily \emph{suggest} that attribution. %...", and nothing else."?
            \item \emph{Events} must be the minimal length necessary to satisfy all other conditions.
            \item The causal relationship identified between the attribution and its associated event must have clearly been intended to have been communicated.
            \item The internality dimension of the attribution must be non-neutral (i.e. either internal or external).
        \end{enumerate}


    \subsubsection{All annotations}
    For \textbf{every annotation} made:
        \begin{enumerate}
            \item Annotations may be non-contiguous.
                \begin{enumerate}
                    \item Use \emph{annotation}\texttt{\_continuation}, replacing \emph{annotation} with the type name of the continued annotation, e.g. \texttt{Event\_continuation}.
                    \item Annotation continuations must not be interrupted by a complete annotation of the same type.
                \end{enumerate}
            \item Annotation spans may fail to provide all relevant information and remain valid only if nearby contextual information relieves the insufficiency.
            \item Avoid annotating sentence-ending punctuation at annotation boundaries.
        \end{enumerate}


    \subsubsection{EAUs}
    \textbf{Every \emph{EAU}}:
        \begin{enumerate}
            \item must consist of the nearest pair of annotations for which both spans satisfy their respective conditions, except:
                \begin{enumerate}
                    \item when an attribution is repeated in such a way that does not seem to be merely sentence repair (e.g. stuttering, clarification, etc.)
                    % TODO: start a "common mistakes" section and put this in it -> \item When the same \emph{event} is mentioned multiple times, for which an \emph{attribution} is given only once, only the nearer text span will be included, \emph{provided} the nearer span does not significantly rely on contextual information to meet it's other conditions.
                \end{enumerate}
            \item must not overlap in its event and attribution spans.
            \item given a contiguous series of similar events which an attribution or series of attributions with similar internality explains, annotate a single event attribution grouping like events as a single event and like attributions as a single attribution.
        \end{enumerate}


    \subsection{Annotation tips}

    \subsubsection{EAUs}
    \begin{description}
        \item[Daisy-chaining] It is acceptable for the same span of text to be selected as the caused \texttt{Event} in one EAU and separately as an \texttt{Attribution} as part of a different EAU.
            \begin{description}
                \item[\emph{Example:}]\mbox{}\\
                \textbf{E:} I just had a thirty minute panic attack.\\ 
                \textbf{A:} My credit card was not where I usually keep it.\\
                \textbf{E:} My credit card was not where I usually keep it.\\
                \textbf{A:} I had changed the location.
            \end{description}
    \end{description}
    \begin{description}
        \item[Concept repetition] It is acceptable to annotate the same conceptual EAU as many times as it is mentioned, but not when it seems only to have been repeated for the sake of conversational clarity.
    \end{description}

    \subsubsection{Attribution Dimensions}
    \begin{enumerate}
    \item Internality
        \begin{description}
            \item[Who is to blame? Who takes the credit?] In answering the question, ``What degree of internality/externality is present in this Attribution?" you will necessarily be answering the question, ``According to the speaker, who/what is to blame (in the case of a negative event) or who/what should take credit (in the case of a positive event) for the associated Event?"
                
            If there seems to be more than just the speaker responsible for a given event, try to identify who has more focus, and adjust annotation accordingly. A heuristic for identifying who has more focus is identifying which of those responsible are mentioned first.
                \begin{description}
                \item[\emph{Example:}]\mbox{}\\
                \textbf{E:} The house looks marvelous.\\ 
                \textbf{A:} I had the cleaning ladies over.\\
                \texttt{Internality:5/7}; ``I" has more focus than ``the cleaning ladies."
                \end{description}
                \begin{description}
                \item[\emph{Example:}]\mbox{}\\
                \textbf{E:} The house looks marvelous.\\ 
                \textbf{A:} The cleaning ladies came over.\\
                \texttt{Internality:3/7}; Though the cleaning ladies are the only ones explicitly mentioned, it is understood within the context that the speaker must have taken some action for them to come over at all, therefore precluding annotation from being \emph{highly} external.  
                \end{description}
                \begin{description}
                \item[\emph{Example:}]\mbox{}\\
                \textbf{E:} I can't keep myself motivated\\ 
                \textbf{A:} because of my depression.\\
                \texttt{Internality:7/7}; The speaker refers to an illness of theirs by means of a personally referring pronoun, which is different from...
                \end{description}
                \begin{description}
                \item[\emph{Example:}]\mbox{}\\
                \textbf{E:} I can't keep myself motivated\\ 
                \textbf{A:} because of the depression.\\
                \texttt{Internality:5/7}; Though the depression is known to belong to the speaker, the speaker chooses not to use any personal reference when mentioning the depression.
                \end{description}
                \begin{description}
                \item[\emph{Example:}]\mbox{}\\
                \textbf{E:} You can't keep yourself motivated\\ 
                \textbf{A:} because of your depression.\\
                \texttt{Internality:5/7}; Assuming that the context suggests that the information relevant to this generic second person mentioned (i.e. the ``\emph(you)'s") is actually highly relevant to the speaker, the speaker chooses not to use any properly personal reference when mentioning the depression.
                \end{description}
        \end{description}

    \item Stability
        \begin{description}
            \item[Temporally constrained Attributions, temporally abstracted Events] When annotating the stability of an Attribution, you are essentially annotating the likelihood that the Attribution will be considered a cause of the Event by the speaker if the Event were to happen again.

                In answering this question, it will be necessary to hypothesize about future occurences of the Event, but be careful not to extend this hypothetical way of thinking to the Attribution itself. That is to say that in answering this question, your mental model should consider the Attribution constrained to the particular instance mentioned in time, but consideration of the Event should be abstracted to all possible future occurences of the Event, regardless of how unlikely it really would be to happen again.
        \end{description}

    \item Globality
        \begin{description}
            \item[Avoid considering butterfly effects] Annotating Globality essentially is answering the question, ``How wide is the scope of things believed to be affected by this Attribution?" Inevitably, some Attributions have little prior dialogue to rely on in answering this, in which case Seligman's guidelines prompt you to mentally model an average person as a surrogate to gauge the Attribution's effects on.

                When doing this, try only to count the obvious immediately following effects and not all of what might be likely chain reactions of each of those effects. For example, if given an Attribution to the effect of ``They're changing my blood pressure medication," the only obvious domain in which that plays a major, immediate role is the speaker's physical health, though a wide range of future changes might possibly or probably change as a result. However, the majority of these further effects would likely only be a result of whatever physical change the medication brings about, and not the change in medication itself, therefore the Attribution ought to be considered more Specific than Global.

                All this isn't to say that an Attribution with an otherwise narrow scope of effect could never be annotated as more pervasive; if you believe that the speaker believes it's pervasive, mark it as such.
        \end{description}
    \end{enumerate}

\end{document}
